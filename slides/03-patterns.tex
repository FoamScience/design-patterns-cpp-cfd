\begin{frame}[fragile]{Design patterns: A quick run-down}

\begin{figure}
\begin{center}
\begin{tikzpicture}[node distance=1.5cm,%
    bigbox/.style={draw, rectangle, minimum width=3.5cm, minimum height=5.5cm, text depth=4.5cm, text=white},%
    smallbox/.style={draw, rectangle, minimum width=2cm, minimum height=0.75cm, fill=white},%
    picst/.style={anchor=west, inner sep=0.1cm, text width=4cm, font=\small},
    picst2/.style={anchor=west, inner sep=0.1cm, text width=4cm, font=\scriptsize, xshift=-0.5cm},
    picst3/.style={anchor=east, inner sep=0.1cm, text width=4cm, font=\scriptsize},
    picst4/.style={anchor=east, inner sep=0.1cm, text width=4cm, font=\small, align=right},
    picst5/.style={anchor=east, inner sep=0.1cm, text width=4cm, font=\scriptsize, xshift=0.5cm, align=right},
    ]
\renewcommand{\umltextcolor}{white}
\renewcommand{\umlfillcolor}{mDarkTeal}
\renewcommand{\umldrawcolor}{white}
% creational
\uncover<1-5,10,15> {
\node (creational) [bigbox,fill=mDarkTeal] at (0,0) {\bf Creational};
\node<1-5,10,15> (builder) [smallbox] at (creational.north) [yshift=-1.5cm] {Builder};
\node<1-5,10,15> (builder-desc) at (builder.north west) {\emoji{building-construction}};
\node<1> (builder-pic) at (builder.center -| creational.east) [picst] {Construct complex\\ objects step by step.};
\node<1> (builder-pic2) at (builder-pic.east) [picst2] 
{
    \begin{tikzpicture}[scale=0.5,node distance=0.5cm, overlay]
    \tikzstyle{every node}=[font=\scriptsize]
    \begin{class}[text width=4cm]{LinkedList<parcel>}{3,3}
        \operation{ctor(Istream\&, fvMesh\&)}
    \end{class}
    \begin{class}[text width=4cm]{parcel}{6.5,-1}
        \attribute{\# mesh : const Foam::fvMesh\&}
        \operation{+ operator>>(Istream\&) : Istream\&}
        \operation{+ clone() : autoPtr<parcel>}
    \end{class}
    \begin{class}[text width=4cm]{iNew}{-3,-4.8}
        \attribute{\# mesh : const Foam::fvMesh\&}
        \operation{+ operator()(Istream\&, fvMesh\&) : autoPtr<parcel>}
    \end{class}
    \renewcommand{\umltextcolor}{mDarkTeal}
    \renewcommand{\umldrawcolor}{mDarkTeal}
    \composition{LinkedList<parcel>}{builder}{0..*}{parcel}
    \unidirectionalAssociation{parcel}{}{\tiny subclass}{iNew}
    \end{tikzpicture}%
};
\renewcommand{\umltextcolor}{white}
\renewcommand{\umlfillcolor}{mDarkTeal}
\renewcommand{\umldrawcolor}{white}
\node<2-5,10,15> (factory) [smallbox] at (builder.north) [yshift=-1.5cm] {Factory Method};
\node<2-5,10,15> (factory-desc) at (factory.north west) {\emoji{factory}};
\node<2> (factory-pic) at (factory.center -| creational.east) [picst] {Create concrete objects through a base class};
\node<2> (factory-pic2) at (factory-pic.east) [picst2] 
{
    \begin{tikzpicture}[scale=0.5,node distance=0.5cm, overlay]
    \tikzstyle{every node}=[font=\scriptsize]
    \begin{abstractclass}[text width=5cm]{nuModel}{5,2}
        \attribute{\# ptrTable : HashTable<nuModel::ctor\*, word>}
        \operation{+ {\bf New()} : autoPtr<nuModel>}
        \operation[0]{+ correct() = 0 : void}
    \end{abstractclass}
    \begin{class}[text width=4cm]{concreteNuModel}{5,-3}
        \operation{+ correct() : void}
    \renewcommand{\umltextcolor}{mDarkTeal}
    \renewcommand{\umldrawcolor}{mDarkTeal}
    \inherit{nuModel}
    \end{class}
    \begin{class}[text width=4cm]{concreteNuModel}{5,-3}
        \operation{+ correct() : void}
    \end{class}
    \draw[dotted] (5, 5) -- (nuModel.north);
    \umlnote ( note ) at (5, 5)
    {
        \mintinline{cpp}{nuModel::New()} looks up the table and calls the correct ctor
    };
    \renewcommand{\umltextcolor}{mDarkTeal}
    \renewcommand{\umldrawcolor}{mDarkTeal}
    \end{tikzpicture}%
};
\node<3-5,10,15> (prototype) [smallbox] at (factory.north) [yshift=-1.5cm] {Prototype};
\node<3-5,10,15> (prototype-desc) at (prototype.north west) [text=gray] {\emoji{robot}$\rightarrow$\emoji{robot}};
\node<3> (prototype-pic) at (prototype.center -| creational.east) [picst] {Clone existing objects};
\node<3> (prototype-pic2) at (prototype-pic.east) [picst2]
{
    \begin{tikzpicture}[scale=0.5,node distance=0.5cm, overlay]
    \tikzstyle{every node}=[font=\scriptsize]
    \begin{class}[text width=4cm]{parcel}{6.5,8}
        \attribute{\# mesh : const Foam::fvMesh\&}
        \operation{+ operator>>(Istream\&) : Istream\&}
        \operation{+ clone() : autoPtr<parcel>}
    \end{class}
    \begin{class}[text width=4cm]{iNew}{6.5,1}
        \attribute{\# mesh : const Foam::fvMesh\&}
        \operation{+ operator()(Istream\&, fvMesh\&) : autoPtr<parcel>}
    \end{class}
    \renewcommand{\umltextcolor}{mDarkTeal}
    \renewcommand{\umldrawcolor}{mDarkTeal}
    %\composition{LinkedList}{builder}{0..*}{parcel}
    \unidirectionalAssociation{parcel}{}{\tiny subclass}{iNew}
    \end{tikzpicture}%
};
\node<4-5,10,15> (singleton) [smallbox] at (prototype.north) [yshift=-1.5cm] {Singleton};
\node<4-5,10,15> (singleton-desc) at (singleton.north west) {\emoji{keycap-1}};
\node<4-4> (singleton-pic) at (singleton.center -| creational.east) [picst] {Exactly one instance!\\ Hidden ctors};
\node<4> (singleton-pic2) at (singleton-pic.east) [picst2] {\hspace{1cm} \mintinline{cpp}{Foam::Time}? not really!};
\node<4> (singleton-pic2) at (singleton-pic.east) [picst2]
{
    \begin{tikzpicture}[scale=0.5,node distance=0.5cm, overlay]
    \tikzstyle{every node}=[font=\scriptsize]
    \begin{class}[text width=4cm]{object}{6.5,5}
        \attribute{\# ctor()}
        \attribute{\# {\bf singleton} : object*}
        \operation{+ {\bf New()} : autoPtr<object>}
    \end{class}
    \draw[dotted] (6.5, 8) -- (object.north);
    \umlnote ( note ) at (6.5, 8)
    {
        \mintinline{cpp}{if (singleton == nullptr)} \\ \mintinline{cpp}{ singleton = new ctor() } };
    \renewcommand{\umltextcolor}{mDarkTeal}
    \renewcommand{\umldrawcolor}{mDarkTeal}
    \end{tikzpicture}%
};
}

\renewcommand{\umltextcolor}{white}
\renewcommand{\umlfillcolor}{mLightGreen}
\renewcommand{\umldrawcolor}{white}

% structural
\uncover<5-10,15> {
\node (structural) [bigbox,fill=mLightGreen] at (4,0) {\bf Structural};
\node<6-10,15> (proxy) [smallbox] at (structural.north) [yshift=-1.5cm] {Proxy};
\node<6-10,15> (proxy-desc) at (proxy.north west) {\emoji{fast-forward}};
\node<6> (proxy-pic) at (proxy.center -| structural.east) [picst] {Provide a substitue or placeholder};
\node<6> (proxy-pic2) at (proxy.center -| structural.west) [picst3] {
    \begin{tikzpicture}[scale=0.5,node distance=0.5cm, overlay]
    \tikzstyle{every node}=[font=\scriptsize]
    \begin{class}[text width=4.5cm]{fvMeshSubsetProxy}{2.5,3}
        \attribute{+ subsettype : enum}
        \operation{+ baseMesh() : const fvMesh\&}
        \operation{+ mesh(): const fvMesh\&}
        \operation{+ interpolate(fvMeshsubset\&, Field\&): tmp<Field>}
    \end{class}
    \draw[dotted,mLightGreen] (2.5, -4) -- (fvMeshSubsetProxy.north);
    \umlnote ( note ) at (2.5, -4)
    { Delegate to interpolation on \\ whole meshes, or subsets of meshes (cell sets and zones)};
    \renewcommand{\umltextcolor}{mLightGreen}
    \renewcommand{\umldrawcolor}{mLightGreen}
    \end{tikzpicture}%
};
\node<7-10,15> (adapter) [smallbox] at (proxy.north) [yshift=-1.5cm] {Adapter};
\node<7-10,15> (adapter-desc) at (adapter.north west) {\emoji{arrows-clockwise}};
\node<7> (adapter-pic) at (adapter.center -| structural.east) [picst] {Collaboration of \\ incompatible interfaces};
\node<7> (adapter-pic2) at (adapter.center -| structural.west) [picst3]
{
    \begin{tikzpicture}[scale=0.5,node distance=0.5cm, overlay]
    \tikzstyle{every node}=[font=\scriptsize]
    \begin{class}[text width=4.5cm]{CPLSocketFOAM}{2.5,5}
        \operation{+ initCFD() : void}
        \operation{+ pack() : void}
        \operation{+ unpack() : void}
    \end{class}
    \begin{class}[text width=4.5cm]{CPLSocketLAMMPS}{2.5,-2}
        \operation{+ initMD() : void}
        \operation{+ pack() : void}
        \operation{+ unpack() : void}
    \end{class}
    \draw[dotted,mLightGreen] (2.5, 0.5) -- (CPLSocketFOAM.south);
    \draw[dotted,mLightGreen] (2.5, 0.5) -- (CPLSocketLAMMPS.north);
    \umlnote ( note ) at (2.5, 0.5)
    { CPL bindings for C++ (originally written in Fortran)};
    \renewcommand{\umltextcolor}{mLightGreen}
    \renewcommand{\umldrawcolor}{mLightGreen}
    \end{tikzpicture}%
};
\node<8-10,15> (facade) [smallbox] at (adapter.north) [yshift=-1.5cm] {Facade};
\node<8-10,15> (facade-desc) at (facade.north west) {\emoji{door}};
\node<8> (facade-pic) at (facade.center -| structural.east) [picst] {Simplified interface to\\ libraries};
\node<8> (facade-pic2) at (facade.center -| structural.west) [picst3]
{
    \begin{tikzpicture}[scale=0.5,node distance=0.5cm, overlay]
    \tikzstyle{every node}=[font=\scriptsize]
    \begin{class}[text width=5cm]{smartRedisClient}{2.5,5}
        \attribute{\# mesh : const fvMesh\& }
        \attribute{\# redisDB : tmp<smartRedisAdapter>}
        \operation{\# updateNamingConvention() : void}
        \operation{+ sendGeometricFields (const wordList\&) : void}
        \operation{+ packFields<T>  (DataSet\&, const wordList\&) : void}
        \operation{+ sendList<Type> (List<Type>\&, const word\&) : void}
    \end{class}
    \renewcommand{\umltextcolor}{mLihghtGreen}
    \renewcommand{\umldrawcolor}{mLightGreen}
    \end{tikzpicture}%
};
\node<9-10,15> (composite) [smallbox] at (facade.north) [yshift=-1.5cm] {Composite};
\node<9-10,15> (composite-desc) at (composite.north west) {\emoji{grapes}};
\node<9> (composite-pic) at (composite.center -| structural.east) [picst] {Compose objects into a tree structure};
\node<9> (composite-pic2) at (composite.center -| structural.west) [picst3]
{
    \begin{tikzpicture}[scale=0.5,node distance=0.5cm, overlay]
    \tikzstyle{every node}=[font=\scriptsize]
    \begin{class}[text width=4cm]{polyMesh}{3,9}
        \attribute{\# points : pointIOField}
        \attribute{\# faces : faceIOList}
        \operation{+ area() : scalar}
    \end{class}
    \begin{class}[text width=2cm]{pointIOField}{5,4}
    \end{class}
    \begin{class}[text width=2cm]{faceIOList}{2,0}
        \operation{+ area() : scalar}
    \end{class}
    \begin{class}[text width=2cm]{face}{-1,5}
        \operation{+ faceArea() : scalar}
    \end{class}
    \renewcommand{\umltextcolor}{mLightGreen}
    \renewcommand{\umldrawcolor}{mLightGreen}
    \composition{polyMesh}{\tiny points}{1}{pointIOField}
    \composition{polyMesh}{\tiny faces}{1}{faceIOList}
    \composition{faceIOList}{}{1..*}{face}
    \end{tikzpicture}%
};
}

\renewcommand{\umltextcolor}{mDarkTeal}
\renewcommand{\umlfillcolor}{mLightBrown}
\renewcommand{\umldrawcolor}{white}

\uncover<10-> {
\node (behavioral) [bigbox,fill=mLightBrown] at (8,0) {\bf Behavioral};
\node<11-15> (iterator) [smallbox] at (behavioral.north) [yshift=-1.5cm] {Iterator};
\node<11-15> (iterator-desc) at (iterator.north west) {\emoji{scroll}};
\node<11> (iterator-pic) at (iterator.center -| behavioral.west) [picst4] {Traverse elements of a collection};
\node<11> (iterator-pic2) at (iterator-pic.west) [picst5]
{
    \begin{tikzpicture}[scale=0.5,node distance=0.5cm, overlay]
    \tikzstyle{every node}=[font=\scriptsize]
    \begin{class}[text width=4cm]{UList<T>}{-4,3}
        \attribute{+ using iterator = T*}
        \attribute{+ using const\_iterator = const T*}
        \attribute{+ using reverse\_iterator = std::reverse\_iterator<iterator>}
        \attribute{+ using const\_reverse\_iterator = std::reverse\_iterator<const\_iterator>}
    \end{class}
    \begin{class}[text width=4cm]{LinkedList<T>}{-5,-4}
    \end{class}
    \begin{class}[text width=2.5cm]{iterator}{4,-3}
        \operation{+ next() : iterator\&}
        \operation{+ prev() : iterator\&}
        \operation{+ good() : bool}
    \end{class}
    \draw[dotted,mLightBrown] (-7,-2.5) -- (LinkedList<T>.north);
    \draw[dotted,mLightBrown] (-7,-2.5) -- (UList<T>.north);
    \umlnote ( note ) at (-7, -2.5)
        {\mintinline{cpp}{for(auto& e: list) {}}};
    \renewcommand{\umltextcolor}{mLightBrown}
    \renewcommand{\umldrawcolor}{mLightBrown}
    \unidirectionalAssociation{LinkedList<T>}{}{\tiny subclass}{iterator}
    \end{tikzpicture}%
};
\node<12-15> (strategy) [smallbox] at (iterator.north) [yshift=-1.5cm] {Strategy};
\node<12-15> (strategy-desc) at (strategy.north west) {\emoji{hammer-and-wrench}};
\node<12> (strategy-pic) at (strategy.center -| behavioral.west) [picst4] {Interchangeable\\ algorithms};
\node<12> (strategy-pic2) at (strategy-pic.west) [picst5]
{
    \begin{tikzpicture}[scale=0.5,node distance=0.5cm, overlay]
    \tikzstyle{every node}=[font=\scriptsize]
    \begin{class}[text width=4cm]{divScheme<T>}{-4,3}
        \attribute{\# interpScheme: \\tmp<interpolationScheme<T> >}
        \operation[0]{+ fvcDiv() : tmp<Field<T> >}
    \end{class}
    \begin{class}[text width=4cm]{gaussDivScheme<T>}{2,-1}
        \operation{+ fvcDiv(Field\&) : tmp<Field<T> >}
        \renewcommand{\umltextcolor}{mLightBrown}
        \renewcommand{\umldrawcolor}{mLightBrown}
        \inherit{divScheme<T>}
    \end{class}
    \begin{class}[text width=4cm]{gaussDivScheme<T>}{2,-1}
        \operation{+ fvcDiv(Field\&) : tmp<Field<T> >}
    \end{class}
    \draw[dotted,mLightBrown] (-7,-2) -- (gaussDivScheme<T>.north);
    \umlnote ( note ) at (-7, -2)
        {does \mintinline{cpp}{surfaceIntegrate(}\\\mintinline{cpp}{interpScheme.dotInterpolate}\\\mintinline{cpp}{(mesh.Sf(), vf))}\\ interpolation to face centers alters how divSchemes behave};
    \renewcommand{\umltextcolor}{mLightBrown}
    \renewcommand{\umldrawcolor}{mLightBrown}
    \end{tikzpicture}%
};
\node<13-15> (observer) [smallbox] at (strategy.north) [yshift=-1.5cm] {Observer};
\node<13-15> (observer-desc) at (observer.north west) {\emoji{loudspeaker}};
\node<13> (observer-pic) at (observer.center -| behavioral.west) [picst4] {Subscription channels \\ event notifications};
\node<13> (observer-pic2) at (observer-pic.west) [picst5]
{
    \begin{tikzpicture}[scale=0.5,node distance=0.5cm, overlay]
    \tikzstyle{every node}=[font=\scriptsize]
    \begin{class}[text width=4cm]{MeshObject}{-5,3}
        \operation{+ {\bf movePoints}(registry\&) : void}
        \operation{+ {\bf updateMesh}\\(registry\&, mapPolyMesh\&) : void}
    \end{class}
    \begin{class}[text width=3cm]{pointMesh}{2,7}
        \renewcommand{\umltextcolor}{mLightBrown}
        \renewcommand{\umldrawcolor}{mLightBrown}
        \inherit{MeshObject}
    \end{class}
    \begin{class}[text width=3cm]{pointMesh}{2,7}
    \end{class}
    \begin{class}[text width=3cm]{faMesh}{2,-1.5}
        \renewcommand{\umltextcolor}{mLightBrown}
        \renewcommand{\umldrawcolor}{mLightBrown}
        \inherit{MeshObject}
    \end{class}
    \begin{class}[text width=3cm]{pointMesh}{2,7}
    \end{class}
    \begin{class}[text width=3cm]{faMesh}{2,-1.5}
    \end{class}
    \begin{class}[text width=3cm]{objectRegistry}{-5,5}
    \end{class}
    \begin{class}[text width=3cm]{dynamicFvMesh}{-8,7.5}
        \operation[0]{+ update() = 0 : void}
        \renewcommand{\umltextcolor}{mLightBrown}
        \renewcommand{\umldrawcolor}{mLightBrown}
        \inherit{objectRegistry}
    \end{class}
    \begin{class}[text width=3cm]{dynamicFvMesh}{-8,7.5}
        \operation[0]{+ update() : void}
    \end{class}
    \draw[dotted,mLightBrown] (-7,-1) -- (MeshObject.north);
    \umlnote ( note ) at (-7, -1)
        {in \mintinline{cpp}{dynamicFvMesh::update()} \\ \mintinline{cpp}{for(auto& obj : regObjs)}\\ \mintinline{cpp}{{obj.updateMesh(mesh, map)}}};
    \renewcommand{\umltextcolor}{mLightBrown}
    \renewcommand{\umldrawcolor}{mLightBrown}
    \unidirectionalAssociation{MeshObject}{\tiny register}{}{objectRegistry}
    \end{tikzpicture}%
};
\node<14-15> (visitor) [smallbox] at (observer.north) [yshift=-1.5cm] {Visitor};
\node<14-15> (visitor-desc) at (visitor.north west) {\emoji{detective}};
\node<14> (visitor-pic) at (visitor.center -| behavioral.west) [picst4] {Seperate Algorithms and objects};
\node<14> (visitor-pic2) at (visitor-pic.west) [picst5]
{
    \begin{tikzpicture}[scale=0.5,node distance=0.5cm, overlay]
    \tikzstyle{every node}=[font=\scriptsize]
    \begin{class}[text width=4cm]{fvSolver<T>}{-3,10}
        \operation{+ solve(const dictionary\&) : SolverPerformance<T>}
    \end{class}
    \begin{class}[text width=5cm]{lduMatrix}{-4,4}
        \attribute{\# preconditioner : lduMatrix::preconditioner}
        \attribute{\# smoother : lduMatrix::smoother}
        \attribute{\# solver : lduMatrix::solver}
    \end{class}
    \begin{class}[text width=3cm]{fvMatrix<T>}{-8,6}
        \renewcommand{\umltextcolor}{mLightBrown}
        \renewcommand{\umldrawcolor}{mLightBrown}
        \inherit{lduMatrix}
    \end{class}
    \begin{class}[text width=3cm]{fvMatrix<T>}{-8,6}
    \end{class}
    \renewcommand{\umltextcolor}{mLightBrown}
    \renewcommand{\umldrawcolor}{mLightBrown}
    \unidirectionalAssociation{lduMatrix}{}{\tiny calls}{fvSolver<T>}
    \unidirectionalAssociation{fvSolver<T>}{}{\tiny subclass}{fvMatrix<T>}
    \end{tikzpicture}%
};
}

\end{tikzpicture}
\end{center}
\vspace{0.5cm}
\caption{Design pattern examples from OpenFOAM (simplified interfaces)}\label{fig:01}
\end{figure}

\end{frame}
