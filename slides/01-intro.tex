\begin{frame}[fragile]{30mins Journey through C++ Programming Paradigms}

    \begin{itemize}
        \item Programming paradigms for CFD/C++ developers
        \item Principles for better software design
        \item Most common design pattern in Open Source C++ CFD software
        \item Applications: Design patterns for common mechanisms
        \item Closing remarks
    \end{itemize}

\end{frame}

\begin{frame}[allowframebreaks,fragile]{Programming pardigms}

    It's all about how  to manage {\bf the program's state}

\begin{description}
    \item[Procedural Programming\hspace{2cm}] \hspace{\linewidth} 
        State mutated externally to code.
\begin{CodeEnvNoComment}[Mutate a variable]{cpp}{\scriptsize}
// Symptoms: pass-by-reference, void return type of free functions
void inc(int &x) { x++; }
int x = 0;
inc(x);
\end{CodeEnvNoComment}
        "Give a man a state, and he will have a bug one day. Teach him to mutate state everywhere, and he will have bugs for a lifetime"
        --- Mutated Joshoa Bloch's saying
    \item[Object-Oriented Programming\hspace{2cm}] \hspace{\linewidth} 
        State and code logic are coupled.
\begin{CodeEnvNoComment}[Object classes]{cpp}{\scriptsize}
// Symptoms: class data members, member methods
class Counter {
    int x = 0;
public:
    void inc() { x++;}
};
Counter i;
i.inc();
\end{CodeEnvNoComment}
    Check \href{https://www.youtube.com/watch?v=F4fxlM5gmtU}{Tomislav's \bf Object-oriented crash course for OpenFOAM devs}
    \item[Functional Programming\hspace{2cm}] \hspace{\linewidth} 
        State, huh? There shall be no state,
        \\only immutable variables and zero-side-effects functions
\begin{CodeEnvNoComment}[No-capture pass-by-value lambdas]{cpp}{\scriptsize}
// Symptoms: pure functions, immutability everywhere
// math-correctness; i.e. cannot write x=x+1
auto inc = [](int x) { return x + 1; };
const int x = 0;
const int xPlusOne = inc(x);
\end{CodeEnvNoComment}
        No scientific computing in Haskell, or F\# - non-existent user base.
    \item[Declarative Programming\hspace{2cm}] \hspace{\linewidth} 
        State management is abstracted away. Facts/Rules/Queries as seen in logic
        programming and Database Systems (SQL ... etc)
\begin{CodeEnvNoComment}[C++ ranges but it's c++23]{cpp}{\scriptsize}
// Symptoms: Queries, rules and facts
// Transform is a view rule, lazy evaluated! but do we care?
// as long as it does what it's supposed to
auto inc = [](int x) { return x + 1; };
vector<int> v = {1, 2, 3};
auto result = v | ranges::views::transform(inc);
\end{CodeEnvNoComment}
        Popular in Database Systems and web stuff; not so much in scientific computing.
    \item[Parallel Programming\hspace{2cm}] \hspace{\linewidth} 
        State can be shared, and must be carefully managed.\\
        Fear of race conditions and deadlocks gives rise to locks, mutexes, and atomic operations.\\
\begin{CodeEnvNoComment}[Execution policies since c++17]{cpp}{\scriptsize}
// Symptoms: multithreading, MPI, GPU offloading
auto inc = [](int x) { return x + 1; };
vector<int> v = {1, 2, 3, 4};
std::transform(std::execution::par, v.begin(), v.end(), v.begin(), inc);
\end{CodeEnvNoComment}
        Check my \href{https://openfoam-parallelisation-course.github.io/}{Workshop: \bf Parallel programming in OpenFOAM}
    \item[Generic Programming\hspace{2cm}] \hspace{\linewidth} 
        State is abstracted away, and code logic is type-agnostic.
\begin{CodeEnvNoComment}[Concepts and templates]{cpp}{\scriptsize}
// Symptoms: meta-programming, templating and compile-time programming
template<typename T>
concept Incrementable = std::is_move_constructible<T>::value
    && requires(T x) { { x + 1 } -> std::convertible_to<T>; };
template<Incrementable T> T inc(T x) { return x + 1; }
int x = 0;
if constexpr (Incrementable<decltype(x)>) {
    x = inc(x);
} else {
    // Something else
}
\end{CodeEnvNoComment}
\end{description}

\end{frame}
