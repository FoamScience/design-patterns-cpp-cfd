\begin{frame}[fragile]{Design patterns for common mechanisms - RTS}

\begin{description}
    \item[Scenario 1\hspace{2cm}] \hspace{\linewidth}
        You want users to select a model for a \underline{particular concern} {\bf at runtime}.\\
        These models are implemented as children of a base (template) class.\\
        You also want them to add new models without altering your code.\\
    \item[Scenario 2\hspace{2cm}] \hspace{\linewidth}
        You have a legacy code base, that you want to unit-test.\\
        But writing a binary for each test is cumbersome.\\
        So the unit tests should be selectable {\bf at runtime} too.\\
    \item<2>[Solution\hspace{2cm}] \hspace{\linewidth}
        A runtime type selection table (RTS) that maps names to objects.\\
        Think of it as a manual {\tt vtable}.\\
        Relying mainly on {\bf Global (or static) variables} and a bit of metaprogramming.\\
\end{description}

\end{frame}

\begin{frame}[fragile]{Design patterns for common mechanisms - RTS}

Here is a little experiment with OpenFOAM's RTS:

\begin{CodeEnvNoComment}[Exploring memory layouts with GDB]{shell}{\scriptsize}
git clone https://github.com/FoamScience/foamUT
cd foamUT
source /usr/lib/openfoam/openfoam2112/etc/bashrc # Or any version
# Compile with debug symbols
sed -i 's/14/14 -g -ggdb -O0/g' tests/exampleTests/Make/options
./Alltest --no-parallel
gdb ./tests/exampleTests/testDriver
    (gdb) b main
    (gdb) r --- -case cases/cavity
    (gdb) ptype 'Foam::Function1<double>'
    # In particular, we are interested in:
    (gdb) ptype 'Foam::Function1<double>::dictionaryConstructorTableType'
    (gdb) ptype 'Foam::Function1<double>::dictionaryConstructorPtr'
    # See what's available through:
    (gdb) ptype 'Foam::Function1Types::CSV<double>::'
    # Here is the metaprogramming part:
    (gdb) ptype 'Foam::Function1<double>::adddictionaryConstructorToTable<*>::'
\end{CodeEnvNoComment}
\end{frame}

\begin{frame}[allowframebreaks,fragile]{Design patterns for common mechanisms - RTS}

Discoveries:

\begin{itemize}
    \item {\bf Base class} keeps a {\bf static pointer} for a {\tt HashTable}
        of {\bf function pointers} to ctors of 'derived' classes.
        \begin{enumerate}
            \item {\bf static} so it gets initialized before main, after dynamic library loading.
            \item {\bf pointer} because order of initialization of global variables is {\bf not guaranteed}.
            \item Technically; ctors don't have addresses, so cannot have function pointers to them.\\
                Instead we store pointers to little helper construction functions.
        \end{enumerate}
    \item {\bf The memory} pointed to by the table pointer is managed manually
        \begin{enumerate}
            \item {\tt \*ConstructorTablePtr\_construct(bool)} called by the base class ctor.
            \item Flexible enough to have multiple ways to construct objects (from dictionary, from Istream, etc)
        \end{enumerate}
    \pagebreak
    \item Instatiation of {\tt Base::add\*ConstructorToTable<Derived>} will cause Derived's ctor to be added to the table.
        \begin{enumerate}
            \item Which is an effect of how template subclasses work.
            \item and with some macros, the boilerplate code is burried.
        \end{enumerate}
    \item The factory pattern comes into play:\\
        \mintinline{cpp}{autoPtr<Base> obj = Base::New(ctor_args);}\\
        which will forward the args to the selected ctor depending on a type name from user configs\\
        No need to \mintinline{cpp}{#include "Derived.H"} and if Derived implements some
        pure virtual functions from Base, they will be called instead.
\end{itemize}
\end{frame}

\begin{frame}[fragile]{Design patterns for common mechanisms - objectRegistry}

\begin{description}
    \item[Scenario 1\hspace{2cm}] \hspace{\linewidth}
        You have a nice RTS for viscosity models; implementing all kinds of fluids.\\
        Base ctor now takes 10 args so you cover evey desirable field to calculate $\nu$\\
        Next guy comes by and wants to use custom fields... and now your interface needs to change!\\
    \item[Scenario 2\hspace{2cm}] \hspace{\linewidth}
        Your ModelA depends on a ModelB from another library, but ModelB also depends on ModelA.\\
        Flat-out bad design, but it's too late to refactor.\\
    \item<2>[Solution\hspace{2cm}] \hspace{\linewidth}
        Promise to write class ctors with {\bf nothing more than (a config + a mesh)} as arguments.\\ 
        Have an Object Registry with 1 rule: Only takes global object in.\\
\end{description}

\end{frame}

\begin{frame}[allowframebreaks,fragile]{Design patterns for common mechanisms - objectRegistry}

Say you want a function object (a UDF) to send fields to a Database for some ML/AI processing:

\begin{CodeEnvNoComment}[]{cpp}{\scriptsize}
// In system/controlDict
functions
{
    SendPUandPhi
    {
        type fieldsToSmartRedis; // gets loaded because of the RTS
        libs ("libsmartredisFunctionObjects.so");
        fields (p U phi); // <- dont care about their types! just send these fields out plz
        patches (internal);
    }
}
\end{CodeEnvNoComment}

How do you think the code will look like?

\pagebreak

Right, should be simple enough

\begin{CodeEnvNoComment}[Trivial implementation; works for like 2 secs]{cpp}{\scriptsize}
// Nice start, maybe a pure function (hopefully no side effects)
void Foam::smartRedisClient::sendGeometricFields
(
    const volScalarField& p,
    const volVectorField& U,
    const surfaceScalarField& phi
) const; // blah blah
\end{CodeEnvNoComment}

But, what happens when someone wants to send temperature?

\pagebreak

Exploit: Fields are registered to the mesh they were created on:

\begin{CodeEnvNoComment}[API improvements]{cpp}{\scriptsize}
void Foam::smartRedisClient::sendGeometricFields
(
    const fvMesh& mesh // less dependencies -> stable API
    // You selfish ppl, this means less compilation time for YOU!
) const
{
    const auto& p = mesh.lookupObject<volScalarField>("p");
    const auto& U = mesh.lookupObject<volVectorField>("U");
    const auto& phi = mesh.lookupObject<surfaceScalarField>("phi");
    // the same blah blah from before
}
\end{CodeEnvNoComment}

Much better, ensured a stable interface, but still needs changes to account for new fields

\pagebreak

Generic programming to the rescue!

\begin{CodeEnvNoComment}[Externally configurable  so more easily testable]{cpp}{\scriptsize}
void Foam::smartRedisClient::sendGeometricFields
(
    const dictionary dict, // now configurable
    const fvMesh& mesh
) const
{
    wordList fields = dict.lookup("fields"); // get fields list from config
    checkFieldsExist<SupportedTypes>(fields, mesh);
    sendFields<SupportedTypes>(fields, mesh);
}
\end{CodeEnvNoComment}

Delegation to templated methods is good, gives control over supported types through type lists.
C++ is, in the end, a typed language.

\pagebreak

\begin{CodeEnvNoComment}[]{cpp}{\scriptsize}
template<class... Types> bool Foam::smartRedisClient::checkFieldsExist (
    const wordList& fieldNames, const objectRegistry& obr
) const {
    // static_assert at least one template argument
    forAll(fieldNames, fi) {
        // Fold expressions to check if a matching name of any of the types is found
        if (!(obr.foundObject<Types>(fieldNames[fi]) || ...)) {
            // Be transparent with the poor user seeing this for the first time
            word supportedTypes = word("(") + nl;
            ((supportedTypes += tab + Types::typeName + nl), ...);
            supportedTypes += ")";
            FatalErrorInFunction
                << "Field " << fieldNames[fi] << " not found in objectRegistry"
                << " as any of the supported types:" << nl
                << supportedTypes
                << exit(FatalError); }
    }
    return true;
}
\end{CodeEnvNoComment}

\end{frame}
