\begin{frame}[allowframebreaks,fragile]{Principles for better software design}

\begin{itemize}
    \item \textbf{Program for interfaces}
    \begin{itemize}
        \item Inheritence (Composition?) and Polymorphism in OOD are your best friends.
        \item Relating to the Facade and Strategy patterns.
        \item Example of API design: Three Levels of API calls in
            \href{https://github.com/OFDataCommittee/openfoam-smartsim/blob/main/docs.md}{OpenFOAM-SmartSim}
            \\(Service, Developer, and Generic interfaces)
        \item Benefits: Modularity, Flexibility, Dependency Injection, and Testability.
        \item Enhanced with generic programming (specifically concepts).
    \end{itemize}
    \pagebreak
    \item \textbf{Seperation for concerns}
    \begin{itemize}
        \item OpenFOAM is built mostly as a set of dynamic libraries linked to a binary.
        \item You can write code for your concerns (new BC? new model?) and load it at runtime.
        \item Relating to the Factory, Strategy and Registry patterns.
        \item Example of implementing Load-balanced adaptive mesh refinement by hooking to the
            dynamic mesh library and extending its classes:
            \href{https://github.com/STFS-TUDa/blastAMR}{blastAMR}
        \item Benefits: Modularity, Parallel Development and easier Maintability.
    \end{itemize}
    \pagebreak
    \item \textbf{Composition Over Inheritance}
    \begin{itemize}
        \item OpenFOAM deviates from this principle significantly.
        \item The overhead of dynamic dispatch and indirection introduced by composition make it so
        the benefits from composition are not worth it! Inheritence plays better with polymorphism
        \item Relating to the Strategy, Decorator, and dependency injection patterns.
        \item Example 1:
            \href{https://github.com/su2code/SU2/blob/master/SU2_CFD/include/fluid/CFluidModel.hpp}{SU2's fluid model class}
            has viscosity, diffusivity, and thermal conductivity as data members instead of inheriting from them.
        \item Example 2:
            \href{https://develop.openfoam.com/Development/openfoam/-/blob/OpenFOAM-v2312/src/OpenFOAM/matrices/schemes/schemesLookup.H}{schemesLookup class from OpenFOAM}
            is composed of different scheme kinds (interpolation, div, grad, ..., etc)
            instead of inheriting from their base classes.
        \item Benefits: Reduced coupling, no fragile base classes. Also, easier on the unit tests.
    \end{itemize}
    \pagebreak
    \item \textbf{Principle of Least Astonishment}
    \begin{itemize}
        \item Aim for interfaces to be intuitive and surprise-free.
        \item Examples:
\begin{CodeEnvNoComment}[Transport equations]{cpp}{\scriptsize}
// How hard it is to figure out the terms?
fvm::ddt(T) + fvm::div(phi, T) == fvm::laplacian(DT, T);
\end{CodeEnvNoComment}
\begin{CodeEnvNoComment}[Ways to find a Max]{cpp}{\scriptsize}
int a, b; Foam::volScalarField c;
Foam::max(a, b);
c.max(); max(c); Foam::gMax(c);
Foam::volScalarField d = c.max(1.0);
\end{CodeEnvNoComment}
        \item Benefits: Less of a learning curve and less missunderstanding bugs.
    \end{itemize}
\end{itemize}


\end{frame}
