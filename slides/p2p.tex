\section{Point-to-Point communication}

\begin{frame}[fragile]{Communicators and ranks}
    There may be many processes talking!
    \begin{description}
        \item[MPI Communicators] \hspace{\linewidth} Objects defining which processes can communicate;
            Processes are refered to by  their {\bf ranks}
        \begin{itemize}
            \item \verb|MPI_COMM_FOAM| in the Foundation version and Foam Extend 5
            \item \verb|MPI_COMM_WORLD| (All processes) elsewhere
            \item Size: \verb|Pstream::nProcs()|
        \end{itemize}
    \item[MPI rank \hspace{1.9cm}] \hspace{\linewidth} Process Identifier (an integer).
        \begin{itemize}
            \item \verb|Pstream::myProcNo()| returns the active process's ID.
        \end{itemize}
    \end{description}
\end{frame}


\begin{frame}[fragile]{P2P comms}

\begin{itemize}
    \item MPI defines its own {\bf Data Types}
    \item OpenFOAM gets around it using {\bf parallel streams}
        \begin{itemize}
            \item OpenFOAM hands over a stream-representation of your data to MPI calls
            \item MPI passes the information in those streams around
        \end{itemize}
\end{itemize}

\begin{figure}
    \begin{center}
    \begin{tikzpicture}
    \node[loris1, process] (p0) [label=above:{\scriptsize $P_0$}] {};
    \node[loris2, process] (p1) at (6,0) [label=above:{\scriptsize $P_1$}] {};
    \node (c) at ($(p0)+(3,-1)$) {   };
    \draw[thick,draw=mDarkTeal,Tee Barb-] (p0) |- (c.west) node [near end,above, mDarkTeal] {\tt IPstream};
    \draw[thick,draw=mLightGreen,-stealth] (c.east) -| (p1) node [near start,above,mLightGreen] {\tt OPstream};

    %\node[node, image] (x_src_i) [right=of y_src_i, label=below:$\phi_j$] { };
    \node (s1) at (3,-1) [semicircle, fill=mLightGreen, minimum size=1mm, rotate=-135, node distance=0,anchor=south] {};
    \node (s1) at (3,-1) [semicircle, fill=mDarkTeal, minimum size=1mm, rotate=45,node distance=0,anchor=south] {};
    \node[draw=mDarkTeal] (b0) at (s1.south) [mDarkTeal,left=.45cm,label=left:{\scriptsize send},rotate=90] {\scriptsize MPI buf};
    \node[draw=mLightGreen] (b1) at (s1.south) [mLightGreen, right=.45cm,label=right:{\scriptsize recv},rotate=-90] {\scriptsize MPI buf};
    \end{tikzpicture}
    \end{center}
    \caption{Communication between two processes in OpenFOAM}
    \label{fig:}
\end{figure}

\end{frame}

\againframe<2>{f1}

\begin{frame}[fragile]{P2P comms: A first example}

    \begin{itemize}
        \item {\tt Pstream} class provides the interface needed for communication
        \item Each "send" must be matched with a "recieve"
    \end{itemize}
\begin{CodeEnvNoComment}[Slaves talk to master]{cpp}{\tiny}
if (Pstream::master())
{
    // Receive lst on master
    for
    (
        int slave=Pstream::firstSlave();
        slave<=Pstream::lastSlave();
        slave++
    )
    {
        labelList lst;
        ?\colorbox{mLightGreen!20}{IPstream fromSlave}?(Pstream::commsTypes::blocking, slave);
        fromSlave >> lst; // Then do something with lst
    }
} else {
    // Send lst to master
    ?\colorbox{mLightGreen!20}{OPstream toMaster}?(Pstream::commsTypes::blocking, Pstream::masterNo());
    toMaster << localLst;
}
\end{CodeEnvNoComment}
\end{frame}

\begin{frame}[fragile]{P2P Blocking comms}

    {\tt Pstream::commsTypes::blocking} (or just {\tt Pstream::blocking} in Foam Extend) defines properties for the MPI call which is executed by
    the constructed stream.
    \begin{itemize}
        \item Does a "local blocking send", i.e. acts on a local buffer
        \item No matching receive available yet? Block until the message is copied into the buffer
        \item Returns when the send buffer is safe to be reused.
        \item Blocking recieve only returns when the receive buffer has the expected data.
        \item Use this if you want to be on the safe side.
        \item But, It may result in deadlocks
    \end{itemize}
\end{frame}

\begin{frame}[fragile]{P2P Blocking comms}

    {\tt Pstream::commsTypes::scheduled} (or just {\tt Pstream::scheduled} in Foam Extend) lets MPI pick the best course of action
    (in terms of performance and memory). This may also depend on the MPI implementation.
    \begin{itemize}
        \item Does a "standard send", Either:
            \begin{enumerate}
                \item The message is directly put in the recieve buffer.
                \item Data is buffered (similar to `blocking`).
                \item Block until a receive shows up.
            \end{enumerate}
        \item Has higher chances of causing deadllocks
    \end{itemize}

    \begin{description}
        \item[A Deadlock] happens when a process is waiting for a message that never reaches it.
    \end{description}
\end{frame}



\begin{frame}[fragile]{P2P Blocking comms: Deadlocks}

    \begin{itemize}
        \item Either a matching send or a recieve is missing (Definitely a deadlock).
        \item A send-recieve cycle (Incorrect usage or order of send/recieve calls).
    \end{itemize}
\begin{figure}

    \begin{center}
    \begin{tikzpicture}
    \node[loris1, process] (p0) [label=above:{\scriptsize $P_0$}] {};
    \node[loris2, process] (p1) at (6,0) [label=above:{\scriptsize $P_1$}] {};
    \node (c) at ($(p0)+(3,-1)$) {   };
    \draw[thick,draw=mDarkTeal,stealth-] (p0) |- (c.west) node [near end,above, mDarkTeal] {};%%{\tt IPstream};
    \draw[thick,draw=mLightGreen,-stealth] (c.east) -| (p1) node [near start,above,mLightGreen] {};%%{\tt OPstream};

    %\node[node, image] (x_src_i) [right=of y_src_i, label=below:$\phi_j$] { };
    \node (s1) at (3,-1) [semicircle, fill=mLightGreen, minimum size=1mm, rotate=-135, node distance=0,anchor=south] {};
    \node (s1) at (3,-1) [semicircle, fill=mDarkTeal, minimum size=1mm, rotate=45,node distance=0,anchor=south] {};
    \node[draw=mDarkTeal] (b0) at (s1.south) [mDarkTeal,left=.45cm,label=left:{\scriptsize recv},rotate=90] {\scriptsize MPI buf1};
    \node[draw=mDarkTeal] (b00) at (b0.east) [mDarkTeal,left=.5cm,label=left:{\scriptsize send},rotate=90] {\scriptsize MPI buf2};
    \node[draw=mLightGreen] (b1) at (s1.south) [mLightGreen, right=.45cm,label=right:{\scriptsize send},rotate=-90] {\scriptsize MPI buf1};
    \node[draw=mLightGreen] (b10) at (b1.west) [mLightGreen, right=.5cm,label=right:{\scriptsize recv},rotate=-90] {\scriptsize MPI buf2};
    \end{tikzpicture}
    \end{center}
    \caption{Deadlock possibility due to a 2-processes send-recieve cycle (Kind of depends on MPI implementation used!).}
    \label{fig:}
\end{figure}
\end{frame}

\begin{frame}[fragile]{P2P Non-Blocking comms}

    {\tt Pstream::commsTypes::nonBlocking} (or just {\tt Pstream::nonBlocking} in Foam Extend) 
    does not wait until buffers are safe to re-use.
    \begin{itemize}
        \item Returns immediately.
        \item The program must wait for the operation to complete ({\tt Pstream::waitRequests}).
        \item It's a form of piepline parallelism; i.e. Overlaps computation and communication.
    \end{itemize}

    \begin{itemize}
        \item Avoids Deadlocks
        \item Minimizes idle time for MPI processes
        \item Helps skip unnecessary synchronisation
    \end{itemize}
\end{frame}

\begin{frame}[fragile]{P2P Non-Blocking comms: An example}
\begin{CodeEnvNoComment}[Communicate with a neighboring processor]{cpp}{\scriptsize}
// Code for the Foundation version and ESI
?\colorbox{mLightGreen!20}{PstreamBuffers pBufs}?(Pstream::commsTypes::nonBlocking);
// Send
forAll(procPatches, patchi)
{
  UOPstream toNeighb(procPatches[patchi].neighbProcNo(), pBufs);
  toNeighb << patchInfo;
}
?\colorbox{mLightGreen!20}{pBufs.finishedSends();}? // <- Calls Pstream::waitRequests
// Receive
forAll(procPatches, patchi)
{
  UIPstream fromNb(procPatches[patchi].neighbProcNo(), pBufs);
  Map<T> nbrPatchInfo(fromNb);
}
\end{CodeEnvNoComment}
\end{frame}

\pgfplotsset{width=7cm,compat=newest}

\begin{frame}[fragile]{Overlapping communication and computation}
\begin{figure}
    \begin{center}
        \scriptsize
        \begin{tikzpicture} [
            every axis/.style={ % add these settings to all the axis environments in the tikzpicture
                /pgf/number format/.cd, fixed, precision=3,1000 sep={}, zerofill,
                every node near coord/.append style={font=\tiny,},
                ybar stacked,
                %%x tick label style={rotate=45,anchor=east},
                symbolic x coords={
                    64, 32, 16, 8
                },
                enlargelimits=0.2,
                xtick=data,
                ymax=0.8,
                ylabel={Total execution time (in secs)},
                xlabel={Message size in Mb},
                ytick={0,0.2,0.4,0.6,0.8},
              bar width=15pt,
	          nodes near coords,
              },
        ]
\begin{axis}[
    ybar stacked,
    hide axis,
    bar shift=-8pt,
	bar width=15pt,
    legend style={at={(0.5,1.25)},
      anchor=north,legend columns=2},
    ]
\addplot[ybar,fill=mDarkTeal,draw=mDarkTeal,text=white] plot coordinates {(64,0.163979) (32,0.0819085) (16,0.0402079) (8,0.0222183)};
\label{nbct}
\addplot[ybar,fill=mLightGreen,draw=mLightGreen,text=white] plot coordinates {(64,0.289514) (32,0.144492) (16,0.0739917) (8,0.036825)};
\label{nbdt}
\end{axis}
\begin{axis}[
    bar shift=8pt,
    legend style={at={(0.5,1.02)},
      anchor=south,legend columns=2},
    ]
\addplot[ybar,fill=mDarkBrown,draw=mDarkBrown,text=white] plot coordinates {(64,0.167) (32,0.084) (16,.0417193) (8,0.0211096)};
\label{bct}
\addplot[ybar,fill=mLightBrown,draw=mLightBrown,text=white] plot coordinates {(64,0.657209) (32,0.26363) (16,0.134693) (8,0.07258)};
\label{bdt}
\addlegendentry{Computation time (Blocking)}
\addlegendentry{Diff. to total time (Blocking)}
\addlegendimage{/pgfplots/refstyle=bdt,color=mDarkTeal}
\addlegendentry{Computation time (NonBlocking)}
\addlegendimage{/pgfplots/refstyle=bdt,color=mLightGreen}
\addlegendentry{Diff. to total time (NonBlocking)}
\end{axis}
\end{tikzpicture}
    \end{center}
    \caption{Effect of message size on overlapping communication and computation (4 processors, OpenMPI 4, OpenFOAM 8); \scriptsize Benchmark inspired from \cite{Baruffa2020}}
\end{figure}
\end{frame}


\begin{frame}[fragile]{MPI send modes used in OpenFOAM code}
\begin{figure}
    \begin{center}
    \begin{tikzpicture}
        \node[minimum height=1cm, minimum width=2cm,draw=white, text=white,fill=mDarkTeal] (m0) {\scriptsize Standard Send};
        \node[note] (n0) at (m0.north east)  {\tiny Default in scheduled};
        \node[lownote] (ln0) at (m0.south west)  {\tiny {\tt MPI\_Send}};
        \node[draw=white, fill=gray!20] (d0) at ($(m0)+(1cm,-1.2cm)$){\scriptsize
                Will not return until send buffer is safe to use,
                May also buffer.};
        \draw[-stealth, draw=mDarkBrown] (ln0.south) -- (d0.north);

        \pause
        \node[minimum height=1cm, minimum width=2cm,draw=white, text=white,fill=mDarkTeal] (m1) at ($(m0)+(6cm,0)$) {\scriptsize Buffered Send};
        \node[note] (n1) at (m1.north east)  {\tiny Default in blocking};
        \node[lownote] (ln1) at (m1.south west)  {\tiny {\tt MPI\_Bsend}};
        \node[draw=white, fill=gray!20] (d1) at ($(m1)+(-1cm,-1.8cm)$){\scriptsize
                Returns immediately and the send buffer can be used};
        \draw[-stealth, draw=mDarkBrown] (ln1.south) -- (d1.north);

        \pause
        \node[minimum height=1cm, minimum width=2cm,draw=white, text=white,fill=mDarkTeal] (m5) at ($(m0)+(3cm,2cm)$) {\scriptsize Non-Blocking standard Send};
        \node[note] (n5) at (m5.north east)  {\tiny Default in nonBlocking};
        \node[lownote] (ln5) at (m5.south west)  {\tiny {\tt MPI\_Isend}};

        \pause
        \node[minimum height=1cm, minimum width=2cm,draw=white, text=white,fill=mDarkTeal] (m2) at ($(m0)+(0,-3cm)$) {\scriptsize Synchronous Send};
        \node[note] (n2) at (m2.north east)  {\tiny not used};
        \node[lownote] (ln2) at (m2.south west)  {\tiny {\tt MPI\_Ssend}};
        \node[minimum height=1cm, minimum width=2cm,draw=white, text=white,fill=mDarkTeal] (m3) at ($(m1)+(0,-3cm)$) {\scriptsize Ready Send};
        \node[note] (n3) at (m3.north east)  {\tiny not used};
        \node[lownote] (ln3) at (m3.south west)  {\tiny {\tt MPI\_Rsend}};

    \end{tikzpicture}
    \end{center}
\end{figure}
\end{frame}
